
\section{Software Transactional Memory} % (fold)
\label{sec:software_transactional_memory}

\textit{Software Transactional Memory}, \textbf{STM}, es un módulo que nos da herramientas simples, pero poderosas, que buscan solucionar los problemas de concurrencia comunes (los mencionamos en el capítulo ~\ref{sec:concurrencia}).

La idea es muy sencilla: ejecutamos un bloque de acciones como una transacción, utilizando la función \textbf{atomically}. Una vez que entramos a ese bloque, otros hilos no pueden ver ninguna modificación hasta que terminemos la ejecución, y tampoco podemos ver modificaciones realizados por otros hilos desde ese bloque. Por esas propiedades decimos que nuestra ejecución está \textit{isolada}.

Cuando salimos de la transacción, puede ocurrir una de las dos posibilidades:

1. Si ningún otro hilo modifica los mismos datos, todas las modificaciones ocurridas durante la transacción serán visibles para los otros threads.

2. En el otro caso, todas las modificaciones hechas se descartan y nuestro bloque de acciones se vuelve a ejecutar automáticamente.

En el siguiente ejemplo sencillo de una cuenta bancaria podemos ver su funcionamiento:

\begin{lstlisting}
import Control.Concurrent.STM

type Account = TVar Int

withdraw :: Account -> Int -> STM ()
withdraw acc amount = do
    bal <- readTVar acc
    writeTVar acc (bal - amount)

deposit :: Account -> Int -> STM ()
deposit acc amount = withdraw acc (- amount)

atomically $ do
    acc <- newTVar (20 :: Int)
    deposit acc 20
    withdraw acc 10
    readTVar acc
\end{lstlisting}


% section software_transactional_memory (end)
