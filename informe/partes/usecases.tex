\section{Aplicaciones} % (fold)
\label{sec:aplicaciones}

\subsection{Haskell en el mundo real} % (fold)
\label{sub:haskell_en_el_mundo_real}

A lo largo del informe describimos las principales características de este lenguaje, las fuimos detallando y dando ejemplos. Ahora nos podemos basar en lo descrito y decir para qué y por qué sirve este lenguaje.

Siendo funcional puro y de tipado estático, el lenguaje es atractivo para los programadores: permite escribir bases de código expresivas y seguras, faciles de mantener. El compilador GHC es una gran ventaja también. El código compilado resultante termina siendo comparable en lo performante con lenguajes de tipado estático imperativos, como C o C++.

Es ideal para sistemas concurrentes: su origen funcional facilita la escritura de código concurrente y con ayuda de herramientas, como STM y con un paralelismo sencillo de programar se vuelve seguro y performante, sin la complejidad habitual para ese tipo de aplicaciones y facil de escalar.

Dado que se basa mucho en la teoría y tiene una muy fuerte base matemática, es muy usado en las aplicaciones científicas. Se utiliza para inteligencia artificial y machine learning, protocolos criptográficos, linguística computacional, modelado bayesiano. Tiene aplicaciones en electrónica, robótica y hasta bioinformática \cite{Usecases01}.

% subsection haskell_en_el_mundo_real (end)


\subsection{Programas escritos en Haskell} % (fold)
\label{sub:programas_escritos_en_haskell}

Adelante listaremos algunos de los casos de aplicaciones y librerias grandes y productivas escritas en Haskell.

\begin{itemize}
  \item \textbf{GHC} mismo es uno de los mejores ejemplos de aplicación escrita en Haskell con aproximadamente 150000 líneas de código Haskell en el compilador y otras tantas en librerias del núcleo. Es libre y de código abierto.
  \\ \url{https://www.haskell.org/ghc/}

  \item \textbf{Cabal}, sistema de estructurado y empaquetado de librerias y programas escritos en Haskell. Es usado en los sistemas unix para la instalación de paquetes de Haskell.\\
  \url{https://www.haskell.org/cabal/}

  \item \textbf{Hackage}, registro central de paquetes de código abierto escritos en Haskell. \\
  \url{https://www.haskell.org/cabal/} \\
  Es interesante por varios motivos: aquí podemos encontrar muchisimos paquetes en Haskell para todo tipo de necesidades, que se pueden instalar con la herramienta \textit{cabal-install} (una de las partes del proyecto mencionado en el item anterior). Y otro motivo es que todo el backend del servidor de Hackage está escrito en Haskell. Su código se puede apreciar acá: \\ \url{https://github.com/haskell/hackage-server}

  \item \textbf{Chordify}, servicio web que transforma música, en forma de archivo o links a servicios más conocidos, como youtube o soundcloud, en acordes. \\ \url{http://chordify.net/}

  \item \textbf{XMonad}, es un manejador de ventanas para \textbf{X} que permite organizar ventanas dinamicamente en \textit{tiles}. \\ \url{http://xmonad.org/}

  \item \textbf{Snap} y Yesod son frameworks para desarrollo web bastante sólidos.
  \\ \url{http://snapframework.com/}
  \\ \url{http://www.yesodweb.com/}

  \item \textbf{FPComplete} son una serie de herramientas online para desarrollo y análisis de algoritmos. \\ url{https://www.fpcomplete.com/}

  \item \textbf{Darcs}, una herramienta de versionado de código. \\ url{http://darcs.net/}

  \end{itemize}


% subsection programas_escritos_en_haskell (end)

% section aplicaciones (end)
