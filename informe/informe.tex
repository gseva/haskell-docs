\documentclass{article}

\usepackage[utf8]{inputenc}
\usepackage[T1]{fontenc}
\usepackage{lmodern}
\usepackage{listings}
\usepackage{url}
\usepackage{color}

\definecolor{lightgray}{rgb}{0.93,0.93,0.93}

\lstloadlanguages{Haskell}

\lstset{ %
  backgroundcolor=\color{lightgray},
  language=Haskell
}

\urlstyle{same}

\title{Haskell}

\begin{document}
  \lstset{language=Haskell}
  \maketitle

  \section{Recursión} % (fold)
\label{sec:recursi_n}

La recursión o recursividad es la forma de especificar una función basandose
en su propia definición. Es una parte muy importante de Haskell. Una función
es recursiva cuando una parte de su definición incluye a la propia función.
Necesita por lo menos un \textit{caso base} que no hace llamado recursivo para
que exista una condición límite.

Este ejemplo muestra una función de factorial recursiva, separando
claramente el caso base.

\begin{lstlisting}
  factorial 0 = 1
  factorial n = n * factorial (n - 1)
\end{lstlisting}

Muchas de las funciones comunes en Haskell se pueden definir de forma
recursiva, por ejemplo el \textit{length}, la función que devuelve el
número de elementos de una lista:

\begin{lstlisting}
  length []     = 0
  length (x:xs) = 1 + length xs
\end{lstlisting}

Recursión es usada para definir casi todas las funciones de manejo de números
y listas, sin embargo en la práctica no es de usarse tan seguido: la
recursividad está abstraida por las funciones de las librerias de Haskell,
permitiendo al programador razonar sus problemas en más alto nivel. Por
ejemplo, la función de factorial de ejemplo escrita anteriormente se puede
definir de la siguiente manera:

\begin{lstlisting}
  factorial n = product [1..n]
\end{lstlisting}

% section recursi_n (end)


\section{Pereza} % (fold)
\label{sec:pereza}

Una característica destacable de Haskell es que es \textit{perezoso} (o
\textit{no estricto}). Esto significa que nada se va a evaluar hasta que sea
directamente necesario - la evaluación queda diferida hasta que el resultado
es requerido por otra computación. El pasaje de parámetros a las funciones se
realiza por necesidad.

El siguiente es un ejemplo de la pereza de Haskell:

\begin{lstlisting}
  let (a, b) = (length [1..5], reverse "hola mundo") in 1 + 1
\end{lstlisting}

Como la expresión después del \textit{in} no utiliza los valores de a y b,
estos quedan sin evaluar, ya que no son necesarios. Tambien pueden no ser
necesarios por completo, o ser \textit{parcialmente} requeridos:

\begin{lstlisting}
  let (a, b) = (length [1..5], reverse "hola mundo")
      'o':ss = b
\end{lstlisting}

Como solo es necesaria la primera letra para concluir el matcheo de
patrones, la evaluación es parcial.

Las funciones en Haskell pueden ser perezosas o estrictas. Podemos analizar
si una función es perezosa o estricta pasandole \lstinline$undefined$ y viendo
si su ejecución falla. Esto se debe a que en Haskell la evaluación forzada del
\lstinline$undefined$ siempre termina en un error.

\begin{lstlisting}
  let (x, y) = undefined in x  -- Error!
  length [undefined, undefined, undefined]
  -- No hay error, length es perezoso
\end{lstlisting}


La evaluación perezosa tiene muchas ventajas, como la reutilización de código,
posibilidad de generar estructuras de datos infinitas y definiciones
circulares, pero su mayor inconveniente es que el uso de memoria se vuelve muy
dificil de predecir, por ejemplo las expresiones \lstinline$3+2 :: Int$ y
\lstinline$5 :: Int$ denotan el mismo valor pero pueden tener diferentes
tamaños en memoria.
% \singlespace

% \singlespace y
% section pereza (end)


  \section{Mónadas} % (fold)
\label{sec:m_nadas}

Las \textbf{mónadas} en Haskell se pueden pensar como descripciones
\textit{componibles} de computaciones. Presentan la posibilidad de separar la
combinación de computaciones de su ejecución y permiten acarrear datos extra
implícitamente en adición al resultado de la computación, que
\textit{se producirá} cuando la mónada sea corrida. De esta manera permiten
suplementar las funcionalidades \textit{puras} con I/O, estado, indeterminismo,
etc.

En terminos del lenguaje una mónada es un tipo parametrizado que es instancia
de la clase \textit{Monad}. Su definición es la siguiente:

\begin{lstlisting}

class Monad m where
    return :: a -> m a
    (>>=) :: m a -> (a -> m b) -> m b
    (>>) :: m a -> m b -> m b

\end{lstlisting}

Podemos ver la mónada como un contenedor para un valor \textbf{a}. La función
\lstinline$return$ se ocupa de poner ese valor adentro de la mónada. Entonces
la función \lstinline$(>>=)$, tambien conocida como \textit{bind}, aplica la
función que se le pasa por parámetro al contenido de la mónada obteniendo como
resultado otra mónada (obviamente la función pasada tiene que tener el tipo
adecuado). Se puede ver como funciona en el siguiente ejemplo:

\begin{lstlisting}

putStrLn "Como te llamas?"
>>= (\_ -> getLine)
>>= (\name -> putStrLn ("Hola, " ++ name ++ "!"))

\end{lstlisting}

El operador \lstinline$(>>=)$ se ocupa de tomar el valor del lado izquierdo
y combinarlo con la función del lado derecho para producir un valor nuevo. El
ejemplo de arriba se puede reescribir con la notación \lstinline$do$, que es un
azucar sintáctico alrededor del operador \textit{bind}:

\begin{lstlisting}

do
   putStrLn "Como te llamas?"
   name <- getLine
   putStrLn ("Hola, " ++ name ++ "!")

\end{lstlisting}

Ese código puede parecer de un lenguaje imperativo, y de hecho lo es: otra
forma de ver las mónadas es pensar que son la abstracción necesaria para
suplementar las funcionalidades que no cuadran adentro del paradigma funcional.

La implementación mas sencilla del \lstinline$(>>=)$. toma el valor del lado
izquierdo, le aplica la función y devuelve el resultado, sin embargo se vuelve
realmente útil cuando esa implementación hace algo extra.

\subsection{Las Leyes de las Mónadas} % (fold)
\label{sub:las_leyes_de_las_m_nadas}

Las mónadas por convención deben cumplir las siguientes leyes:

\begin{lstlisting}
  -- Identidad por la izquierda
  return x >>= f = f x

  -- Identidad por la derecha
  m >>= return = m

  -- Asociatividad
  (m >>= f) >>= g = m >>= (x -> f x >>= g)
\end{lstlisting}

% subsection las_leyes_de_las_m_nadas (end)

\subsection{Mónadas comunes} % (fold)
\label{sub:m_nadas_comunes}

La siguiente tabla lista las mónadas más comunes usadas en Haskell, denotando
el problema que tratan de solucionar en términos imperativos.
\linebreak
\begin{tabular}{ | p {3cm} | p {5cm} |}
  \hline
  Mónada & Semántica imperativa \\
  \hline
  \hline
  Maybe & Excepción anónima \\
  \hline
  Error & Excepción con descripción \\
  \hline
  State & Estado global \\
  \hline
  IO & Entrada y Salida \\
  \hline
  [] (list) & Indeterminismo \\
  \hline
  Reader & Entorno \\
  \hline
  Writer & Logger \\
  \hline
\end{tabular}
% subsection m_nadas_comunes (end)

% section m_nadas (end)


  

\subsection{Concurrencia} % (fold)
\label{sub:concurrencia}


Un programa concurrente necesita realizar varias tareas al mismo tiempo. Estas
tareas no necesariamente tienen que estar relacionadas entre si. El correcto
funcionamiento de un programa concurrente no necesita varios núcleos.

En contraste, un programa paralelo soluciona un solo problema con el mejor
rendimiento posible, empleando para eso más de un núcleo.

\subsubsection{Threads} % (fold)
\label{ssub:threads}

Un hilo es una acción \textit{IO} que se ejecuta independientemente
de los otros hilos. Los hilos en Haskell no son determinísticos.
Para crear un thread, usamos la función \textit{forkIO} del módulo
\textit{Control.Concurrent}.

Un ejemplo de uso podría ser la compresión de un archivo

% subsubsection threads (end)

% subsection concurrencia (end)


\subsection{Paralelismo} % (fold)
\label{sub:paralelismo}



% subsection paralelismo (end)




\begin{thebibliography}{3}

\bibitem{Common01}
  ¡Aprende Haskell por el bien de todos!, http://aprendehaskell.es

\bibitem{Common02}
  Real World Haskell, http://book.realworldhaskell.org

\bibitem{Recursion01}
  Recursion, \url{http://en.wikibooks.org/wiki/Haskell/Recursion}

\bibitem{Recursion02}
  Recursion Patterns,
  \url{https://www.fpcomplete.com/school/starting-with-haskell/introduction-to-haskell/3-recursion-patterns-polymorphism-and-the-prelude}

\bibitem{Laziness01}
  Laziness, \url{http://en.wikibooks.org/wiki/Haskell/Laziness}

\bibitem{Monads01}
  Understanding Monads, \url{http://en.wikibooks.org/wiki/Haskell/Understanding_monads}

\bibitem{Monads02}
  Monad, \url{https://wiki.haskell.org/Monad}

\bibitem{Monads03}
  What is a monad?, \url{http://stackoverflow.com/questions/44965/what-is-a-monad}

\bibitem{Monads04}
  Monad laws, \url{https://wiki.haskell.org/Monad_laws}

\bibitem{Monads05}
  All about monads, \url{https://wiki.haskell.org/All_About_Monads}

\bibitem{Monads05}
  The State Monad: A Tutorial for the Confused?, \url{http://brandon.si/code/the-state-monad-a-tutorial-for-the-confused/}

\end{thebibliography}

\end{document}
