\documentclass{article}

\usepackage[utf8]{inputenc}
\usepackage[T1]{fontenc}
\usepackage{lmodern}
\usepackage{graphicx}
\usepackage{listings}
\usepackage{url}
\usepackage{color}

\definecolor{lightgray}{rgb}{0.93,0.93,0.93}

\lstloadlanguages{Haskell}

\graphicspath{ {images/} }

\lstset{ %
  backgroundcolor=\color{lightgray},
  language=Haskell
}

\urlstyle{same}

\title{Haskell}

\begin{document}
  \lstset{language=Haskell}
  \maketitle

  \section{Inferencia de tipos} % (fold)
\label{sec:inferencia_de_tipos}

% section inferencia_de_tipos (end)

Haskell permite definir funciones sin decir de que tipos son estas, esto, muy lejos de querer decir que Haskell no usa un tipado fuerte, es debido a que realiza inferencia de tipos.
Un puede escribir una función que sume dos números como:


\begin{lstlisting}
  sumar  ::  (Num a) => a -> a -> a
  sumar x y = x + y
\end{lstlisting}

Donde la primer linea indica que las variables a son de la clase de tipos Num.
Si en GHCI usamos el comando :t para pedir el tipo de la funcion sumar tendremos:


\begin{lstlisting}
  :t sumar
  sumar  ::  (Num a) => a -> a -> a
\end{lstlisting}

Este resultado era evidente, ya que en la propia definición lo habíamos indicado. Pero Haskell permite definir la función sumar de la siguiente manera:

\begin{lstlisting}
  sumar x y = x + y
\end{lstlisting}

Al utilizar el comando :t en GHCI obtendremos la misma respuesta que antes, aunque esta vez no le  hayamos indicado al lenguaje los tipos en la definición.

\subsection{Algoritmo de Hindley-Milner} % (fold)

% subsection algoritmo_de_hindley_milner (end)
Este resultado proviene de que Haskell realiza inferencia de tipos basandose en el algoritmo de Hindley Milner o Damas Milner.
Este algoritmo sigue seis reglas lógicas:
\begin{figure}[H]
\includegraphics[width=80mm]{hindley-milner}
\end{figure}

Para explicar que quieren decir estas seis reglas primer es necesario entender la sintaxis y el significado de alguno de los símbolos.

Lo primero es la barra horizontal, esta viene a cumplir el rol de una implicación lógica, o sea, si el antecedente, lo de arriba, es verdadero entonces, necesariamente, el consecuente, lo de abajo, también lo es. Cuando en la parte superior hay N hipótesis, para asegurar el consecuente es necesario que se cumplan las N hipótesis, es como si las separaciones por espacio estubiesen indicando una operación and lógica.
Lo siguiente son los dos puntos \":\", estos indican de que tipo es una expresión, por ejemplo x : Int indica que x es del tipo int. El contexto se representa con $\gamma$. Para indicar que algo se encuentra en un determinado contexto usamos $\in$. $\vdash$ es básicamente que se puede demostrar algo, lo precede el contexto necesario para demostrar lo que quiere. La coma \",\" es utilizada para ampliar el contexto, una unión, si entendemos al contexto como el conjunto de referencias a variables de determinados tipos, donde se pisa el elemento si este ya se encontraba en el contexto original. Por ultimo, el símbolo $\sqsubseteq$ indica una especie de inclusión, en realidad es mas bien una herencia, donde el tipo de la izquierda es un subtipo del de la derecha.
Con todos la notación clara es sencillo entender las seis reglas.
Por Ejemplo, la segunda indica que si en nuestro contexto tenemos una expresión e0 que  toma un elemento del tipo  $\tau$ y devuelve otro que es del tipo $\tau$', y en ese mismo entorno tenemos una segunda expresión que es del tipo $\tau$, entonces podemos decir que en ese mismo entorno que la expresión e0 e1 es del tipo $\tau$'.
La quinta nos dice que si en un entorno la expresión e es del tipo $\sigma$' y que $\sigma$' es un subtipo de $\sigma$ entonces la expresión e es del tipo  $\sigma$.

El algoritmo de inferencia de tipos usa estas reglas de forma recursiva. Realiza un ajuste de patrones, donde mira la expresión que tiene, y en cual de los consecuentes encaja mejor, una vez encontrado va a los antecedentes y mira que necesita para que estos se cumplen, si con la información actual es suficiente entonces ha terminado de inducir los tipos, en caso contrario toma estos antecedentes que aun no tiene determinados y vuelve a realizar el mismo ajuste de patrón, y así va infiriendo en cada sub-expresión los tipos necesarios. Cuando encuentra que la información alcanzada es suficiente para cumplir TODAS las hipótesis la inferencia esta completa.

\label{sub:algoritmo_de_hindley_milner}

\subsection{Ejemplo de aplicación del algoritmo} % (fold)
\label{sub:ejemplo_de_aplicaci_n_del_algoritmo}

Si hacemos el seguimiento del como funciona este algoritmo en la función sumar que tenemos más arriba, y utilizando la currificación de Haskell tenemos que la función es de la forma

\begin{lstlisting}
  sumar x y = x + y
\end{lstlisting}

que es lo mismo que

\begin{lstlisting}
  sumar x y = (+) x y
\end{lstlisting}

para ajustar el patrón tomamos

\begin{lstlisting}
  e0 = (+) x
  e1 = y
\end{lstlisting}

esto encaja en el segundo patrón, donde se realiza e0 e1 que es del tipo $\tau$', para esto se induce que e0 es del tipo $\tau$ $\rightarrow$ $\tau$' y que e1 es tipo $\tau$'. Hasta ahora solo sabemos que y es del tipo $\tau$.
Luego hay que ajustar el patrón e0. Donde volvemos a caer en el segundo caso. Entonces tenemos que :

\begin{lstlisting}
  e'0 = (+)
  e'1 = x
\end{lstlisting}

Con e'0 e'1 es del tipo $\tau$ $\rightarrow$ $\tau$'
Concluimos que, dado que e'0 es del tipo $\tau$1 $\rightarrow$ t1' y que e'1 es del tipo $\tau$1, como e'0 e'1 = e0
entonces e0 : $\tau$'1 = $\tau$ $\rightarrow$  $\tau$'
Hasta aquí tenemos que


\begin{lstlisting}[mathescape]
y : $\tau$
x : $\tau$1
(+) :  $\tau$1 $\rightarrow$  $\tau$ $\rightarrow$  $\tau$'
\end{lstlisting}

Además, como (+) : (Num a) => a $\rightarrow$ a $\rightarrow$ a
Por la primera regla tenemos que $\tau$1 $\rightarrow$  $\tau$ $\rightarrow$  $\tau$' = a $\rightarrow$ a $\rightarrow$ a
O sea

\begin{lstlisting}[mathescape]
$\tau$1 = a
$\tau$ = a
t' = a
\end{lstlisting}

Y, finalmente, usando la quinta regla tenemos que

\begin{lstlisting}[mathescape]
$\tau$1 $\sqsubseteq$ a
$\tau$ $\sqsubseteq$ a
t' $\sqsubseteq$ a
\end{lstlisting}

Finalmente solo resta recomponer todo, usando que sumar = e'0 e'1 e1
El tipo de sumar es (Num a) => a $\rightarrow$ a $\rightarrow$ a.

% subsection ejemplo_de_aplicaci_n_del_algoritmo (end)


  \section{Recursión} % (fold)
\label{sec:recursi_n}

La recursión o recursividad es la forma de especificar una función basandose
en su propia definición. Es una parte muy importante de Haskell. Una función
es recursiva cuando una parte de su definición incluye a la propia función.
Necesita por lo menos un \textit{caso base} que no hace llamado recursivo para
que exista una condición límite.

Este ejemplo muestra una función de factorial recursiva, separando
claramente el caso base.

\begin{lstlisting}
  factorial 0 = 1
  factorial n = n * factorial (n - 1)
\end{lstlisting}

Muchas de las funciones comunes en Haskell se pueden definir de forma
recursiva, por ejemplo el \textit{length}, la función que devuelve el
número de elementos de una lista:

\begin{lstlisting}
  length []     = 0
  length (x:xs) = 1 + length xs
\end{lstlisting}

Recursión es usada para definir casi todas las funciones de manejo de números
y listas, sin embargo en la práctica no es de usarse tan seguido: la
recursividad está abstraida por las funciones de las librerias de Haskell,
permitiendo al programador razonar sus problemas en más alto nivel. Por
ejemplo, la función de factorial de ejemplo escrita anteriormente se puede
definir de la siguiente manera:

\begin{lstlisting}
  factorial n = product [1..n]
\end{lstlisting}

% section recursi_n (end)


\section{Pereza} % (fold)
\label{sec:pereza}

Una característica destacable de Haskell es que es \textit{perezoso} (o
\textit{no estricto}). Esto significa que nada se va a evaluar hasta que sea
directamente necesario - la evaluación queda diferida hasta que el resultado
es requerido por otra computación. El pasaje de parámetros a las funciones se
realiza por necesidad.

El siguiente es un ejemplo de la pereza de Haskell:

\begin{lstlisting}
  let (a, b) = (length [1..5], reverse "hola mundo") in 1 + 1
\end{lstlisting}

Como la expresión después del \textit{in} no utiliza los valores de a y b,
estos quedan sin evaluar, ya que no son necesarios. Tambien pueden no ser
necesarios por completo, o ser \textit{parcialmente} requeridos:

\begin{lstlisting}
  let (a, b) = (length [1..5], reverse "hola mundo")
      'o':ss = b
\end{lstlisting}

Como solo es necesaria la primera letra para concluir el matcheo de
patrones, la evaluación es parcial.

Las funciones en Haskell pueden ser perezosas o estrictas. Podemos analizar
si una función es perezosa o estricta pasandole \lstinline$undefined$ y viendo
si su ejecución falla. Esto se debe a que en Haskell la evaluación forzada del
\lstinline$undefined$ siempre termina en un error.

\begin{lstlisting}
  let (x, y) = undefined in x  -- Error!
  length [undefined, undefined, undefined]
  -- No hay error, length es perezoso
\end{lstlisting}


La evaluación perezosa tiene muchas ventajas, como la reutilización de código,
posibilidad de generar estructuras de datos infinitas y definiciones
circulares, pero su mayor inconveniente es que el uso de memoria se vuelve muy
dificil de predecir, por ejemplo las expresiones \lstinline$3+2 :: Int$ y
\lstinline$5 :: Int$ denotan el mismo valor pero pueden tener diferentes
tamaños en memoria.
% \singlespace

% \singlespace y
% section pereza (end)


  
\section{Garbage Collection} % (fold)
\label{sec:garbage_collection}

Muchos lenguajes implementan un Garbage Collection (GC) para ocuparse de los objetos no alcanzables y liberar dicha memoria.


Los lenguajes imperativos suelen generar una cantidad de basura a lo largo de su ejecución, y si estos constan de un GC simplifican la vida del programador, al desligarlo de la necesidad de liberar el espacio no utilizado, y de no deber perder nunca la referencia. De lo contrario, si un lenguaje posee un GC, el programador debe olvidarse de lo que no necesite, simplemente haciendolo inalcanzable, de este modo las referencias almacenadas son solos las que en ese momento resultan útiles, el GC pasará en algún momento y liberará la memoria no alcanzable.


Los lenguajes funcionales, como lo es Haskell, suelen generar muchísima mas basura que los lenguajes imperativos, esto es debido a la inmutabilidad de sus variables, y para guardar un nuevo valor es necesario crear una nueva variable. Cada iteración en un llamado recursivo genera un nuevo valor. En Haskell no es insólito producir 1Gb de datos en un segundo.
Para esto el compilador de Haskell GHC tiene un potente GC que se ocupa de gestionar la basura en forma eficiente.


La inmutabilidad de las variables en Haskell no solo obliga a generar gran cantidad de basura, sino que esta característica es aprovechada para realizar la recolección.
Para esto utiliza el criterio de generación: datos jóvenes y viejos. Las variables mas viejas, al ser inmutables, no apuntan nunca a un dato joven, pues los datos jóvenes no existen en el momento en que las variables viejas son creadas. Esta es la idea que utiliza el recolector de basura de Haskell para aumentar la eficiencia.
El recolector de basura no mirará toda la memoria en la que estuvo trabajando nuestro programa, sino que sólo revisará entre los valores más jóvenes, y liberará los que no estén señalados, que suelen ser la gran mayoría, dado el comportamiento recursivo. Esto también es una ventaja, porque en realidad, mientras mas basura joven aparezca menor es el trabajo que realiza el recolector de basura; esto, que resulta tan poco intuitivo se explica con la forma que se utiliza para almacenar los datos jóvenes y los viejos.
Los datos jóvenes son almacenados en un bloque de memoria especial, una “guardería”, cuando esta guardería se llena el GC mira sólo en esta memoria quienes son alcanzables, por lo tanto útiles y los copia en la memoria de las variables mas viejas, luego nos habilita a reutilizar la “guardería”, la cual se encuentra “vacía”, ya que cualquier dato que pisemos no será útil, o lo tendremos copiado con los valores mas viejos. Este es el motivo por el cual, con mayor cantidad de basura joven, la recolección es más rápida, lo que sucede es que hay menos datos que copiar.

% section garbage_collection (end)


  
\section{Funciones de orden superior} % (fold)
\label{sec:functiones_de_orden_superior}


Una función de orden superior es aquella que puede tomar funciones como parámetro, o devolver una función como resultado, o ambas cosas.

Haskell no solo soporta las funciones de orden superior, sino que hace un uso permanente de esta cualidad, y de forma muy natural.


\subsection{Funciones como parametro} % (fold)
\label{sub:funciones_como_parametro}

Funciones como parámetro

Una función que puede tomar como parámetro a otra función es considerada de orden superior. En Haskell esto se utiliza todo el tiempo para filtrar datos mediante algún criterio, realizar una acción sobre un conjunto de datos, etc.

Un ejemplo de una función que viene por defecto en Haskell que toma funciones como parámetro es la función filter, a la cual se le pasa una función y una lista, filter llama a la función con cada elemento de la lista, si la función devuelve true el elemento es añadido a la lista que da como resultado.

Si se implementa una función fQuickSort a la que le paso una función f que dado un elemento de la lista me devuelve un valor comparable, y una lista que quiero ordenar de la siguiente forma:

\begin{lstlisting}
fQuickSort :: (Ord b) => (a -> b) -> [a] -> [a]
fQuickSort _ [] = []
fQuickSort f (x:xs) =
  let
      menores = fQuickSort f [a | a <- xs, (f a) <= (f x)]
      mayores = fQuickSort f [a | a <- xs, (f a) > (f x)]
  in menores ++ [x] ++ mayores
\end{lstlisting}

En esta implementación es realmente versátil, ya que dependiendo que función f se utilice para medir el “tamaño” de los elementos de la lista obtendremos un resultado distinto, y no restringimos los elementos de la lista a elementos ordenables, dado que el criterio de orden los obtenemos en función de la “medida” de los elementos.

% subsection funciones_como_parametro (end)
\subsection{Funciones como resultado} % (fold)
\label{sub:funciones_como_resultado}

Dijimos que una función también es de orden superior si puede dar como resultado otra función. Esto es útil, pues se puede crear constructores de funciones, estos constructores recibirán un valor, y devolverán una función para cada valor que tomen. Por ejemplo:

\begin{lstlisting}
mutiplicarPor :: (Num a =>) a -> (a -> a)
multiplicarPor x = (*) x
\end{lstlisting}

Para cada valor de x, multiplicarPor da como resultado una función que recibe un numero y devuelve el el producto del numero pasado con x.



En realidad, Haskell utiliza las funciones de orden superior todo el tiempo, ya que las funciones de Haskell solo pueden tomar una única variable, esto es lo mismo que decir que las funciones en Haskell están Currificadas. No hay ninguna contradicción, cuando tenemos una función que aparenta recibir más de una variable lo que en realidad tenemos es una función que recibe un dato y nos devuelve una función, la cual toma el siguiente dato, y nos devuelve otra función y así sucesivamente. Esta es la explicación del porque cuando anotamos en la definición de tipos de una función no diferenciamos entre los parámetros de entrada y el valor de retorno.

La Currificación hace evidente la necesidad de que el lenguaje soporte funciones de orden superior. A medida que vamos aplicando parcialmente la función vamos obteniendo nuevas funciones como resultado, y este es el concepto de que una función sea de orden superior por devolver una función. Usando el ejemplo anterior, el fQuickSort es un constructor de sorts, al que se le pasa una función f y nos da un sort que ordena con un determinado criterio, lo mismo pasa con filter y map.

En Haskell este tipo de funciones son tan comunes que tiene montones de aplicaciones en las librerías standar utilizando funciones de orden superior, ya mencionamos las funciones  filter y map, pero se incluyen muchos más.

% subsection funciones_como_resultado (end)

\subsection{Plieges (folds)} % (fold)
\label{sub:plieges_}

Unas funciones particularmente útiles y cómodas son los pliegues (folds).
Como en Haskell las variables son inmutables no existen los iteradores clásicos de los lenguajes imperativos, sino que se utiliza la recursividad de las funciones, esto es tan común que existen algunas funciones útiles tienen incorporado estos patrones para realizar iteraciones. Si querríamos implementar la función elem utilizando pliegues podríamos hacerlo como

\begin{lstlisting}
elem' :: (Eq a)=> a -> [a] -> Bool
elem' y ys = foldl f False ys
  where f acc x = if x==y then True else acc
\end{lstlisting}

Lo que estamos haciendo es pasarle a nuestro pliegue una función que devuelve True si el elemento que recibe es el mismo que el que nosotros buscamos, y sino devuelve el acc, además la función recibe como “segundo” parámetro el valor inicial del acumulador que va a pasarle a la función, y una lista sobre la que deseamos que itere. Lo que hace foldl es agarrar la cabeza de la lista, pasarselo a la función f, y tomar el valor obtenido como nuevo acumulador, luego se llama a si mismo con la cola de la lista, la misma función y este nuevo acumulador. El resultado que devuelve foldl es el ultimo valor que devuelve el acumulador.

Si quisiéramos implementar foldl podríamos hacerlo así:

\begin{lstlisting}
foldl' :: (a->b->a)-> a->[b]->a
foldl' _ acc [] = acc
foldl' f acc (x:xs) = foldl' f (f acc x) xs
\end{lstlisting}

% subsection plieges_ (end)


\subsection{Funciones Anónimas (lambdas)} % (fold)
\label{sub:funciones_an_nimas_}


Al usar este tipo de funciones solemos tener que crear funciones con el único objetivo de pasarlas a nuestras funciones de orden superior, lo cual no tiene mucho sentido, para esto aparecen las denominadas funciones anónimas, o funciones lambdas. Las funciones lambda son expresiones (devuelven un valor), por eso podemos pasarlas como parámetros a funciones de orden superior.
Su sintaxis es:

\begin{lstlisting}
(\a b -> 2a/b)
\end{lstlisting}

donde a y b son los parámetros que recibe, lo que sucede a  -> indica que el comportamiento de dicha función. Suelen estar encerradas entre paréntesis para delimitarlas, de lo contrario tomarán todo el renglón.
Utilizando este tipo de funciones, podríamos haber escrito nuestro pliegue como

\begin{lstlisting}
elem' :: (Eq a)=> a -> [a] -> Bool
elem' y ys = foldl (\acc x-> if x==y then True else acc) False ys
\end{lstlisting}

% subsection funciones_an_nimas_ (end)

% section functiones_de_orden_superior (end)


  \section{Functores} % (fold)
\label{sec:functores}

Los functores son una clase de tipos que tiene un método llamado fmap, donde

\begin{lstlisting}
fmap :: (a -> b) -> f a -> f b
\end{lstlisting}

La idea de fmap es que toma una funcion de a en b, un functor que contiene un a, y devuelve un functor que tiene un b. Una idea intuitiva de un functor podría ser como una caja, fmap abre esta caja, aplica f al contenido y devulve una caja con el resultado de aplicar la función.

Un término más correcto para definir lo que es un funtor sería contexto computacional. El contexto sería que la computación podría tener un valor, o podría fallar. Si queremos un constructor de tipos que sea una instancia de functor este debe pertenecer a la familia \lstinline$* -> *$.

Algunos functores básicos son las listas [], Maybe, Either a (en este ultimo caso, como necesitamos que nuestro tipo tome un único tipo concreto y Either toma dos debe estar parcialmente evaluda), si se definiese una clase Tree a también podría ser un funtor. Otros no tan claros son IO y (->) r.

Las acciones IO son como cajas que encierran datos que provienen del mundo real, o que saldrán al mismo, podemos ligar el contenido de una acción IO a una variable usando <-, trabajar con estos datos, realizar algunos calculos, pero cuando querramos sacarlos al mundo exterior es necesario transformarlos en otra acción IO usando return. Si queremos mapear un valor IO  lo que en realidad buscamos es obtener una nueva acción IO que contenga el resultado de aplicar una determinada función, esto será:

\begin{lstlisting}
instance Functor IO where
  fmap f action = do
    result <- action
    return (f result)
\end{lstlisting}

El tipo de una función r -> a se puede reescribir como (->) r a esto indica que (->) r es una caja de a. Como -> es de la familia \lstinline$* ->  * -> *$ debe estar parcialmente aplicado, igual que Either.

\begin{lstlisting}
instance Functor ((->) r) where
  fmap f g = (\x -> f (g x))
\end{lstlisting}

Aplicar fmap a una función nos da una nueva función, si pensamos que aplicamos esta nueva funcion a un valor esto daría por resultado la función original aplicada a nuestro valor, y a este resultado le aplicaría la función que le pasamos a fmap, esto no es otra cosa que realizar la composición de funciones.

\begin{lstlisting}
instance Functor ((->) r) where
  fmap = (.)
\end{lstlisting}

\subsection{Leyes de los Functores} % (fold)
\label{sub:leyes_de_los_functores}


Ahora vamos a ver las leyes de los functores. Para que algo sea una functor, debe satisfacer una serie de leyes. Se espera que todos los functores exhiban una serie de propiedades y comportamientos. Deben comportarse fielmente como cosas que se puedan mapear. Al llamar fmap sobre un functor solo debe mapear una función sobre ese funtor, nada más. Este comportamiento se describe en las leyes de los functores. Hay dos de ellas que todas las instancias deFunctor deben cumplir, pero Haskell no hace esta comprobación, debe ser verificada por el usuario que la implementa.

La primera ley de funtores establece que si mapeamos la función identidad sobre un functor, el functor que obtenemos debe ser igual que el original (fmap id = id).

La segunda ley dice que si mapeamos el resultado de una composición de dos funciones sobre un funtor debe devolver lo mismo que si mapeamos una de estas funciones sobre el funtor inicial y luego mapeamos la otra función \lstinline$((f . g) = fmap f)$

% subsection leyes_de_los_functores (end)

\subsection{Functores Aplicativos} % (fold)
\label{sub:functores_aplicativos}


Dentro de un functor podemos tener almacenado cualquier tipo de valor, y como en Haskell las funciones son ciudadanos de primer orden estas también son un valor, por ende podemos tener funciones encerradas dentro de un functor, por ejemplo, si hacemos

\begin{lstlisting}
fmap (*) Just 7
\end{lstlisting}

obtenemos

\begin{lstlisting}
Just (* 7)
\end{lstlisting}

Si queremos aplicar la función almacenada a el valor almacenado por Just 11 podriamos tratar de hacerlo, pero si queremos un comportamiento que funcione en todos los functores debemos crear algo más general, este es el motivo por el cual aparecen los functores aplicativos.

\begin{lstlisting}
class (Functor f) => Applicative f where
  pure :: a -> f a
  (<*>) :: f (a -> b) -> f a -> f b
\end{lstlisting}

La definición de esta clase de tipos nos dice que si alguien es un Applicativa también es un Functor.
Las funciones no están implementadas por defecto, pero pure toma un valor y lo encapsula dentro de un functor aplicativo.
La segunda funcion es una ampliacion de fmap, toma un functor aplicativo, “extrae” su función, y mapea dicha función sobre otro functor. La clase de tipos maybe los implementa de la sigueinte forma:

\begin{lstlisting}
instance Applicative Maybe where
  pure = Just
  Nothing <*> _ = Nothing
  (Just f) <*> something = fmap f something
\end{lstlisting}

% subsection functores_aplicativos (end)

% section functores (end)


  \section{Mónadas} % (fold)
\label{sec:m_nadas}

Las \textbf{mónadas} en Haskell se pueden pensar como descripciones
\textit{componibles} de computaciones. Presentan la posibilidad de separar la
combinación de computaciones de su ejecución y permiten acarrear datos extra
implícitamente en adición al resultado de la computación, que
\textit{se producirá} cuando la mónada sea corrida. De esta manera permiten
suplementar las funcionalidades \textit{puras} con I/O, estado, indeterminismo,
etc.

En terminos del lenguaje una mónada es un tipo parametrizado que es instancia
de la clase \textit{Monad}. Su definición es la siguiente:

\begin{lstlisting}

class Monad m where
    return :: a -> m a
    (>>=) :: m a -> (a -> m b) -> m b
    (>>) :: m a -> m b -> m b

\end{lstlisting}

Podemos ver la mónada como un contenedor para un valor \textbf{a}. La función
\lstinline$return$ se ocupa de poner ese valor adentro de la mónada. Entonces
la función \lstinline$(>>=)$, tambien conocida como \textit{bind}, aplica la
función que se le pasa por parámetro al contenido de la mónada obteniendo como
resultado otra mónada (obviamente la función pasada tiene que tener el tipo
adecuado). Se puede ver como funciona en el siguiente ejemplo:

\begin{lstlisting}

putStrLn "Como te llamas?"
>>= (\_ -> getLine)
>>= (\name -> putStrLn ("Hola, " ++ name ++ "!"))

\end{lstlisting}

El operador \lstinline$(>>=)$ se ocupa de tomar el valor del lado izquierdo
y combinarlo con la función del lado derecho para producir un valor nuevo. El
ejemplo de arriba se puede reescribir con la notación \lstinline$do$, que es un
azucar sintáctico alrededor del operador \textit{bind}:

\begin{lstlisting}

do
   putStrLn "Como te llamas?"
   name <- getLine
   putStrLn ("Hola, " ++ name ++ "!")

\end{lstlisting}

Ese código puede parecer de un lenguaje imperativo, y de hecho lo es: otra
forma de ver las mónadas es pensar que son la abstracción necesaria para
suplementar las funcionalidades que no cuadran adentro del paradigma funcional.

La implementación mas sencilla del \lstinline$(>>=)$. toma el valor del lado
izquierdo, le aplica la función y devuelve el resultado, sin embargo se vuelve
realmente útil cuando esa implementación hace algo extra.

\subsection{Las Leyes de las Mónadas} % (fold)
\label{sub:las_leyes_de_las_m_nadas}

Las mónadas por convención deben cumplir las siguientes leyes:

\begin{lstlisting}
  -- Identidad por la izquierda
  return x >>= f = f x

  -- Identidad por la derecha
  m >>= return = m

  -- Asociatividad
  (m >>= f) >>= g = m >>= (x -> f x >>= g)
\end{lstlisting}

% subsection las_leyes_de_las_m_nadas (end)

\subsection{Mónadas comunes} % (fold)
\label{sub:m_nadas_comunes}

La siguiente tabla lista las mónadas más comunes usadas en Haskell, denotando
el problema que tratan de solucionar en términos imperativos.
\linebreak
\begin{tabular}{ | p {3cm} | p {5cm} |}
  \hline
  Mónada & Semántica imperativa \\
  \hline
  \hline
  Maybe & Excepción anónima \\
  \hline
  Error & Excepción con descripción \\
  \hline
  State & Estado global \\
  \hline
  IO & Entrada y Salida \\
  \hline
  [] (list) & Indeterminismo \\
  \hline
  Reader & Entorno \\
  \hline
  Writer & Logger \\
  \hline
\end{tabular}
% subsection m_nadas_comunes (end)

% section m_nadas (end)


  

\subsection{Concurrencia} % (fold)
\label{sub:concurrencia}


Un programa concurrente necesita realizar varias tareas al mismo tiempo. Estas
tareas no necesariamente tienen que estar relacionadas entre si. El correcto
funcionamiento de un programa concurrente no necesita varios núcleos.

En contraste, un programa paralelo soluciona un solo problema con el mejor
rendimiento posible, empleando para eso más de un núcleo.

\subsubsection{Threads} % (fold)
\label{ssub:threads}

Un hilo es una acción \textit{IO} que se ejecuta independientemente
de los otros hilos. Los hilos en Haskell no son determinísticos.
Para crear un thread, usamos la función \textit{forkIO} del módulo
\textit{Control.Concurrent}.

Un ejemplo de uso podría ser la compresión de un archivo

% subsubsection threads (end)

% subsection concurrencia (end)


\subsection{Paralelismo} % (fold)
\label{sub:paralelismo}



% subsection paralelismo (end)




\begin{thebibliography}{3}

\bibitem{Common01}
  ¡Aprende Haskell por el bien de todos!, http://aprendehaskell.es

\bibitem{Common02}
  Real World Haskell, http://book.realworldhaskell.org

\bibitem{Inference02}
  “What part of Milner-Hindley do you not understand?”, http://stackoverflow.com/questions/12532552/what-part-of-milner-hindley-do-you-not-understand

\bibitem{Recursion01}
  Recursion, \url{http://en.wikibooks.org/wiki/Haskell/Recursion}

\bibitem{Recursion02}
  Recursion Patterns,
  \url{https://www.fpcomplete.com/school/starting-with-haskell/introduction-to-haskell/3-recursion-patterns-polymorphism-and-the-prelude}

\bibitem{GC01}
  GHC Memory Management, \url{https://wiki.haskell.org/GHC/Memory_Management}

\bibitem{Laziness01}
  Laziness, \url{http://en.wikibooks.org/wiki/Haskell/Laziness}

\bibitem{Monads01}
  Understanding Monads, \url{http://en.wikibooks.org/wiki/Haskell/Understanding_monads}

\bibitem{Monads02}
  Monad, \url{https://wiki.haskell.org/Monad}

\bibitem{Monads03}
  What is a monad?, \url{http://stackoverflow.com/questions/44965/what-is-a-monad}

\bibitem{Monads04}
  Monad laws, \url{https://wiki.haskell.org/Monad_laws}

\bibitem{Monads05}
  All about monads, \url{https://wiki.haskell.org/All_About_Monads}

\bibitem{Monads06}
  The State Monad: A Tutorial for the Confused?, \url{http://brandon.si/code/the-state-monad-a-tutorial-for-the-confused/}

\end{thebibliography}

\end{document}
